\chapter*{Závěr}
\addcontentsline{toc}{chapter}{Závěr}

Práce představuje problematiku osobních dat a digitální stopy, ukazuje její technické a ekonomické souvislosti a současné povědomí o~tomto tématu.

Dále rozebírá obecná i konkrétní rizika spojená s~touto problematikou a nabízí několik možných pohledů na problematiku, které pomáhají lépe pochopit provázanost světa z~pohledu soukromí a jeho narušování.

Na to navazuje kategorizací a popisem typů dat, která uživatel svou činností v~digitálním prostředí vytváří, a v~kontextu těchto pohledů a vědomostí nabízí konkrétní postupy a nástroje, které může uživatel ve svém životě používat a tím zvýšit kontrolu nad svou digitální stopou.

Tyto vědomosti pak práce přenáší do části, kde se zabývá tématem z~pohledu českého školského vzdělávání, ukazuje zvyšující se důležitost toho tématu ve výuce, a nabízí východiska pro vznik vzdělávací aplikace.
Prototyp této aplikace, včetně popisu jeho technického řešení, otestování a formulování doporučení dalšího rozvoje, je pak poslední částí práce.
