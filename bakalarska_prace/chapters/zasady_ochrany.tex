\chapter{Klíčové zásady uživatelské ochrany osobních dat}

Je možné se dohadovat, zda má být větší ochrana osobních dat a regulace jejich sběru právně vynucená. V~posledních letech je tento právní tlak na poskytovatele digitálních služeb vyvíjen, v~EU pomocí GDPR, v~USA se tímto směrem vydává California Consumer Privacy Act (CCPA).
Ovšem bez ohledu na to má a bude mít každý uživatel alespoň omezenou možnost kontroly nad svými daty a jejich ochranou.\\
Tato kapitola nabízí pohledy a nástroje, které může uživatel použít pro regulaci vlastního soukromí v~digitálním světě.

\section{Obecné náhledy}
Kromě přímých nástrojů a postupů je dobré si osvojit několik základních pohledů na osobní data a nakládání s~nimi.

\subsection{Co je na internetu, to je veřejné}
Informace, které člověk zveřejní pomocí digitálních platforem (sociální sítě, chatovací aplikace a další), sice působí, že jsou sdílené jen s~omezenou skupinou lidí, je však nutné si uvědomit, že je z~jejich podstaty možné je snadno sdílet i mimo tento okruh. Opakovaně se to dělo, ať chybou v~systému, nebo cíleným sdílením dat.

U~informací, které dává uživatel na internet, je tedy třeba myslet na potenciální zveřejnění dat a rizika s~tím spojená.

% \subsubsection*{Úniky dat}

\subsection{Propojenost}
Ač může mít uživatel pocit, že nemá důvod se o~své soukromí starat, je dobré si uvědomovat, že vytvářením a zveřejňováním dat poskytuje data pro zpřesňování modelů a nepřímo ovlivňuje i další uživatele. Zároveň v~některých situacích, kdy může mít pocit, že sdílí pouze svoje osobní data (například lokační data), v~kombinaci s~daty jiné osoby může vznikat úplně nová informace (například o~tom, kdo se s~kým setkává), kterou uživatel přímo nesdílel. A~dopad na ostatní osoby je jasný ve chvíli, kdy přímo sdílí informace o~někom jiném (například fotografie, kontakty a adresy).  

\subsection{Budoucnost}
Kromě toho je třeba myslet na to, že informace, které nyní jako uživatel vytvářím, budou existovat potenciálně navždy a mohou se vrátit v~budoucím životě -- například při hledání zaměstnání nebo při vstupu do veřejného života. 
Dále i data, která v~tuto chvíli nejsou strojově zpracovatelná či analyzovatelná, v~budoucnu (s~postupující technologií analýzy dat) být mohou.


\section{Zásady a nástroje}

\subsection{Omezení vědomého sdílení informací}
Zejména sociální sítě vedou své uživatele k~tomu, aby sdíleli mnoho informací ze svého života. Tyto informace však mohou uživatele ohrozit, například zvýšeným rizikem vloupání, krádeže identity, cíleného útoku (\textit{social engineering}, \textit{spear phishing}), nebo sdílením těchto informací někam, kde je nechce (viz kapitolu \textit{Rizikové dopady sběru osobních dat}).

Základní zásadou tedy je být si vědom toho, jaké informace vědomě sdílím, včetně vědomí, že se mohou dostat kamkoli a kdykoli.

\subsection{Pravidla pro aplikace a služby}
\subsubsection*{Souhlasy cookies}

Pokud chce uživatel omezit množství dalších dat, která vznikají jeho běžným chováním a která jsou propojitelná do jednoho celku (profilu), jednou z~prvních možností je dbát na možnosti nastavení cookies a sběru dat (které jsou v~EU díky GDPR větší než jinde ve světě). Stránka je povinna uživateli nabídnout možnosti nastavení cookies, včetně odmítnutí těch, které nejsou esenciální pro funkcionalitu služby.
Bohužel je však tato možnost stále často příliš složitá či nesrozumitelná a mnoho uživatelů povolí vše. Jen některé služby nabízí snadné odmítnutí všech neesenciálních cookies.\\

Uložené cookies je také možné z~počítače smazat, obvykle v~nastavení prohlížeče. Některé prohlížeče zaměřené na zvýšené soukromí dále umožňují jak omezit prvotní ukládání (zejména cookies třetích stran), tak nastavit jejich automatické mazání po zavření okna nebo po určité době.

\subsubsection*{Možnost smazání dat}
Také je možné u~služeb vyžádat mazání dat. Na toto mají opět nárok uživatelé v~EU díky GDPR a služby mají povinnost tuto možnost snadno nabízet. Často však jde o~mazání celého účtu a není možné snadno smazat jen část dat. Případně to služby umožňují, ale běžný uživatel o~tom nemusí ani vědět. Existují služby, které toto mohou za uživatele dělat automaticky, například aplikace \href{https://www.jumboprivacy.com}{Jumbo}. 

\subsubsection*{Do Not Track}
S~webovým požadavkem je možné poslat signál \textit{Do Not Track}. Aby se tak dělo, je třeba toto nastavení zapnout v~možnostech prohlížeče. Tento signál dává službám najevo, že uživatel nechce, aby byla jeho aktivita sledována. Aktuálně však není žádná zákonná povinnost ho jakkoli zpracovat, ani neexistuje konsensus na tom, jak by služby, které toto přání uživatele respektovat chtějí, měly postupovat. To se snad v~budoucnu změní, neboť probíhají snahy tento postup sjednotit a dát mu zákonnou váhu \citep{do-not-track-future}.

\subsubsection*{Práva pro aplikace v~mobilních zařízeních}
Velká část činností se přesunula do mobilních zařízení, s~čímž se otevřely i nové možnosti zásahu do soukromí. Aplikace mají možnost přístupu k~různým funkcím zařízení (lokace, mikrofon, gyroskop), uživatel tedy musí dbát na to, jaká práva aplikacím dá, a všímat si, jestli aplikace nepožadují více dat, než by se dalo s~ohledem na jejich účel čekat.

\subsection{Anonymní prohlížení webu}

\subsubsection*{Anonymní režim}
Snad všechny běženě používané prohlížeče nabízejí takzvaný \textit{anonymní režim}. Je důležité si být vědom toho, co tento režim dělá a co ne.
Základním principem je, že soubory cookie a celkový kontext prohlížení uchovává pouze po dobu, kdy je dané okno anonymního prohlížení otevřeno - po zavření okna jsou všechny cookies smazány. To ovšem uživatele nechrání před ostatními typy sběru dat a identifikace - dříve zmíněný \textit{browser fingerprinting}, \textit{behavioral profiling} nebo \textit{tracking pixel} budou fungovat dále. Stejně tak není nijak pozměněno, jaké informace se dostávají k~poskytovateli internetu.\\

Bohužel ani na část ochrany soukromí, kterou by anonymní režim poskytovat měl, se nelze spolehnout. Opakovaně byly objeveny způsoby, jak díky chybám software sledovat uživatele používající anonymní režim, například pomocí cachování ikon stránek \citep{incognito-tracking}.

\subsubsection*{Prohlížeče}
Jak již bylo zmíněno, různé prohlížeče nabízejí různou míru základního nastavení soukromí i různé možnosti pro uživatelské nastavení.
V~posledních letech vznikají prohlížeče, které této problematice dávají větší důraz -- za zmínku stojí třeba \href{https://vivaldi.com}{Vivaldi} nebo \href{https://brave.com}{Brave}. Dlouhodobě je také velkým advokátem soukromí Mozzila Foundation s~jejich prohlížečem Firefox.
Na druhé straně stojí prohlížeč Google Chrome, který jako součást ekosystému služeb Google defaultně sbírá větší objem dat.


Specifickým případem prohlížeče je Tor Browser (případně další prohlížeče používající stejnou technologii), který umožňuje silně šifrovanou komunikaci, tedy chrání před sledováním ze strany poskytovatele internetu, a stejně jako VPN umožňuje obejít omezení dané například státem, ve kterém se uživatel nachází.  

\subsubsection*{VPN}
VPN (\textit{virtual private network -- virtuální soukromá síť}) je technologií, jež umožňuje připojení ke službám skrze jiné servery, anonymizuje tedy IP adresu uživatele a s~ní spojené informace (zejména lokaci). Je tedy možné je používat pro obcházení geografických omezení, či rizika sledování poskytovatelem (stejně jako u~sítě Tor).

\subsubsection*{Doplňky}
Proti některým konkrétním sledovacím praktikám se dá bránit specializovanými nástroji, dostupnými buď jako samostatné programy, nebo často jako doplňky do prohlížečů.
Lze se takto bránit proti \textit{canvas fingerprinting}u pomocí doplňků jako je \href{https://chrome.google.com/webstore/detail/canvas-blocker-fingerprin/nomnklagbgmgghhjidfhnoelnjfndfpd?hl=en}{Canvas Blocker} nebo \href{https://chrome.google.com/webstore/detail/canvas-fingerprint-defend/lanfdkkpgfjfdikkncbnojekcppdebfp?hl=en}{Canvas Fingerprint Defender}, blokovat \textit{tracking pixel}y v~Gmailu pomocí doplňku \href{https://chrome.google.com/webstore/detail/pixelblock/jmpmfcjnflbcoidlgapblgpgbilinlem?hl=en}{PixelBlock} nebo e-mailových služeb se zabudovanou funkcí (například \href{https://hey.com}{HEY}), i proti \textit{behavioral profiling}u pomocí doplňku přidávajícímu náhodnou prodlevu ke stiskům kláves.
Kromě toho existuje mnoho doplňků, kombinujících více technik ochrany soukromí a blokace obsahu - mezi nejznámější patří AdBlock+, uBlock, Ghostery nebo Privacy Badger.

\subsubsection*{Vyhledávací enginy}
Pokud se chceme vyhnout sběru dat při používání vyhledávačů, je třeba se poohlédnout po alternativách k~nejpoužívanějším vyhledáváčům, jimiž jsou Google, Bing a v~Česku Seznam. Variantou, která neuchovává historii a nevytváří profil uživatele (a tedy nemůže nabízet personalizované výsledky), je vyhledávač DuckDuckGo.

\subsection{Finanční transakce}
Je možné, i když komplikované, chránit svoje soukromí i v~případě finančních transakcí. Běžné finanční operace probíhají přes banky, které naši identitu znají, možnost anonymní výměny peněz však přinesly v~posledních letech kryptoměny. Ty umožňují založení anonymní peněženky, na kterou je možné převést peníze z~běžné měny, a pak převádět peníze mezi peněženkami bez odhalení identity osoby.


\subsection{Šifrování dat}
Dalším způsobem omezení možností práce s~daty, která vytváříme, je jejich šifrování. To se samozřejmě vztahuje jen na část možných digitálních služeb, zejména na komunikaci.
\subsubsection*{Šifrovaná komunikace}
Jak u~chatů, tak u~e-mailu lze najít varianty, kdy je komunikace šifrovaná. To znamená, že je chráněna před čtením třetí stranou. Je důležité si být vědom, že jsou různé způsoby šifrování, a jen některé zaručují, že komunikaci opravdu nemůže číst nikdo kromě komunikujících (tedy ani daná služba). Dále je podstatné, že ač tuto možnost některé služby (například WhatsApp) umožňují, šifrování neprobíhá vždy, ani nemusí být nastaveno jako základní možnost.

Nejznámější chatovací služby zaměřené na šifrovanou komunikaci jsou \href{https://signal.org/}{Signal} a \href{https://telegram.org}{Telegram}.

Šifrovanou komunikaci je možné použít i u~e-mailové služby, zde je nejznámější službou \href{https://protonmail.com}{ProtonMail}. 


\subsection{Zabezpečení dat proti útočníkům}
Ač se tím dostáváme více do oblasti bezpečnosti než soukromí, je nezbytné zmínit i základní zásady ochrany před útočníky, neboť i to může být způsob, jak se naše osobní data mohou dostat do neoprávněných rukou a být zneužita.

\subsubsection*{Ochrana zařízení}
Primárně je důležité -- a zároveň relativně jednoduché -- chránit svá zařízení před útoky. Dva hlavní způsoby jsou:

\textbf{Udržování aktuálního softwaru}\\
Jedním z~častých důvodu průniku do zařízení je neaktualizovaný software, který obsahuje známé bezpečnostní chyby. Typickým případem může být používání již nepodporované verze operačního systému nebo stará verze prohlížeče. Moderní operační systémy i prohlížeče aktualizace vyhledávají samy a upozorňují na ně uživatele (případně je nainstalují bez nutnosti akce uživatelem), u~starších systémů a již nevyvíjených programů to tak nemusí být.

\textbf{Antiviry}\\
Zvýšenou ochranu uživatel získá použitím specializovaného software na odhalování rizikového kódu, například ve stažených souborech (což je jeden z~častých způsobu napadení zařízení). Nabídka antivirových a podobných bezpečnostích programů je široká, zároveň novější verze rozšířeného operačního systému Microsoft Windows obsahují i zabudovaný nástroj Microsoft Defender Antivirus.

\subsubsection*{Phishing}
Počátek phishingových útoků se datuje do devadesátých let. Jde o~konkrétní formu útoků typu \textit{social engineering}, tedy typu útoků, který cílí na uživatele a skrze něj se snaží prolomit zabezpečení. Phishing probíhá typicky formou e-mailů, které se vydávají za důvěryhodné osoby nebo instituce a chtějí po uživateli, aby stáhl soubor či použil odkaz. Phishingové kampaně jsou stále oblíbenou formou útoků -- v~roce 2020 nahlásilo 75 \% firem, že zaregistrovaly phishingový útok, a v~57 \% proběhl alespoň jeden úspěšně \citep{phishing-success}.\\
Ochrana proti těmto útokům nemá jednoduché technické řešení, ale lze dát několik doporučení, jak phishingový útok rozpoznat:
\begin{itemize}
	\item Firmy nebudou požadovat osobní data (heslo, číslo OP, číslo bankovní karty) v~e-mailu ani po telefonu.
	\item Obecně je vhodné nestahovat soubory a neotevírat odkazy z~e-mailu (ale ne vždy se tomu lze vyhnout).
	\item Při otevírání odkazu z~e-mailu je vždy vhodné zkontrolovat, zda adresa opravdu odpovídá tomu, co očekáváme.
	\item Je bezpečnější zkopírovat adresu z~e-mailu a vložit ji ručně do prolížeče, než kliknout na odkaz v~e-mailu.
	\item Phishingové e-maily častěji obsahují gramatické nebo technické chyby.
	\item Phishingové e-maily častěji tlačí příjemce k~rychlé akci (hrozí sankcemi, upozorňují na smyšlená rizika).
\end{itemize}


% \subsubsection*{Šifrované ukládání dat}
% TODO?


\subsubsection*{Hesla a zabezpečení}
Důležitou částí ochrany svých účtů, a tedy dat, která jsou v~nich k~dispozici, je dostatečné zabezpečení přihlašování. Téma hesel by vydalo na samostatnou práci, zmíněno tu bude alespoň několik základních zásad a pravidel.

% \subsubsection*{Úniky hesel}
% TODO?

\textbf{Bezpečné heslo}\\
Je důležité zvolit takové heslo, které je dostatečně obtížné prolomit. Útoky směřující k~prolamování hesel obvykle využívají dvou technik (nebo jejich kombinace) -- \textit{brute force}, kdy útočník zkouší náhodné kombinace znaků, a \textit{slovníkový útok}, který využívá známá slova (případně hesla známá z~předchozích úniků).

To vede ke třem základním doporučením při tvorbě hesel:

\begin{itemize}
	\item \textbf{Heslo by mělo být dostatečně dlouhé}\\
	Tím je lépe chráněno před \textit{brute force} útokem, protože počet kombinací u~delších retězců je takový, že již není prakticky možné vyzkoušet všechny. Doporučovaná délka se mění s~růstem výkonu počítačů i s~pokrokem hashovacích algoritmů, které jsou pro uchovávání hesel používány, hrubé doporučení může být kolem 10 znaků (pokud používáme všechny ASCII znaky) až 25 znaků (pokud používáme pouze velká a malá písmena).

	\item \textbf{Heslo musí být náhodné} (respektive pseudonáhodné)\\
	Pokud je heslo celé nebo z~většiny existujícím slovem (větou, souslovím), výrazně to snižuje jeho bezpečnost. Heslo by mělo být náhodné -- buď řetězec náhodných znaků, nebo (lépe zapamatovatelné) kombinace více náhodně vybraných slov.

	\item \textbf{Heslo musí být unikátní}\\
	Pokud používáme stejné heslo pro více účtů, při úniku hesla je ohrožení větší. Je tedy zásadní mít odlišná hesla v~různých službách. 

\end{itemize}

\textbf{Správci hesel}\\
Naplnit předchozí tři zásady a zároveň si pamatovat hesla k~desítkám až stovkám svých účtů není reálné. Proto velká část uživatelů tyto zásady nedodržuje (i v~případě, že si jich jsou vědomi). Bezpečnostní experti proto obvykle doporučují používat nějakou formu správce hesel -- místo, kde jsou hesla zapsána, ideálně v~šifrované podobě pod hlavním heslem/klíčem.

Různé správce hesel již nabízejí v~základní výbavě internetové prohlížeče, u~nich však není vždy míra zabezpečení největší. Proto jsou i specializované nástroje, které typicky nabízejí větší zabezpečení a větší škálu možností.

\textbf{Vícefaktorová autentizace}\\
Dalším způsobem, jak výrazně zvýšit zabezpečení svého účtu, je vícefaktorová autentizace. Ta obecně znamená, že pro přihlášení je třeba více než jedna věc (což by bylo obvykle heslo). Faktory, které jsou pro přihlášení použité, se dají rozdělit do následujících částí:

\begin{itemize}
	\item \textbf{znalost} -- \textit{něco, co vím}\\ 
		Jde o~heslo, PIN, případně o~odpověď na bezpečnostní otázku.
	\item \textbf{vlastnictví} -- \textit{něco, co mám}\\
		To může být speciální bezpečnostní klíč, ale v~současnosti běžně také mobilní zařízení. 
	\item \textbf{biometrie} -- \textit{něco, co jsem}\\
		Se zlevňováním potřebných senzorů a technologií je běžnou součástí zařízení detektor otisku prstu, kamera a software schopné rozpoznat obličej, a rozpoznávání hlasu. Specializovaná zařízení se pak dají použít například pro kontrolu podle duhovky. Sem lze také zařadit dříve zmíněný \textit{behavioral profiling} použitý pro autentizaci.
	\item \textbf{lokace} -- \textit{kde se nacházím}\\
		V~některých případech lze jako další faktor používat lokaci, a to buď vázanou na konkrétní IP adresy (například zařízení v~kanceláři), nebo (skrze mobilní zařízení) reálnou lokaci.  
\end{itemize} 

Mnoho služeb nabízí dvoufaktorovou autentizaci (2FA), obvykle v~kombinaci hesla a zařízení (použití mobilního zařízení jako klíče při přihlašování na jiném zařízení) nebo biometrie (při přihlašování na zařízení, které tuto formu nabízejí).

Některé služby dokonce tuto míru zabezpečení vyžadují -- ať už ze zákona (služby pro komunikaci s~úřady, jako je v~česku Datová schránka), nebo svým zaměřením na  bezpečnost (e-mailová služba HEY).

% \subsubsection*{Bezpečnostní otázky}
% TODO?

\section*{Shrnutí}
V~této kapitole byly detailně rozebrány konkrétní přístupy a nástroje, které uživateli mohou pomoci zvýšit kontrolu nad digitální stopou a zabezpečit data.
Ukázala, že možností kontroly je více a jejich použití vychází z~potřeb a požadavků každého uživatele.