\chapter{Zařazení problematiky digitální stopy do školních výukových materiálů}

\section{Úvod a vymezení pojmů}

Cílem této kapitoly je dát téma osobních dat a digitální stopy do kontextu vzdělávacích materiálů a východisek, na to, jakou roli a prostor v~nich aktuálně (v~kontextu narůstající důležitosti tématu ve společnosti a společenských debatách) zaujímá. 

Pro snazší orientaci v~kapitole zde shrnuji použité zkratky a kódy a jejich význam:

\textbf{RVP} - Rámcový vzdělávací plán, definován \citep{rvp} je následovně:

\begin{displayquote}
Rámcové vzdělávací programy (RVP) tvoří obecně závazný rámec pro tvorbu školních vzdělávacích programů škol všech oborů vzdělání v~předškolním, základním, základním uměleckém, jazykovém a středním vzdělávání. Do vzdělávání v~České republice byly zavedeny zákonem č. 561/2004 Sb., o~předškolním, základním, středním, vyšším odborném a jiném vzdělávání (školský zákon).
\end{displayquote}

\textbf{Klasifikace SŠ} (dle \citep{stredni-vzdelavani}):

\textbf{Obory s~maturitou}

\textbf{SŠ (M)} - úplné střední odborné vzdělání s~maturitou (obory kategorie M)

\begin{displayquote}
příprava má profesní charakter a délka studia je 4 roky. Po maturitě lze pokračovat ve vzdělávání na vysoké nebo vyšší odborné škole.
\end{displayquote}

\textbf{SŠ (L)} - úplné střední odborné vzdělání s~odborným výcvikem a maturitou (obory kategorie L)

\begin{displayquote}
studium připravuje pro náročná dělnická povolání a nižší řídicí funkce. V~denní formě je 4leté a jeho významnou součástí je odborný výcvik (obory vznikly z~dřívějších 3letých učebních oborů). Absolventi získávají maturitní vysvědčení a mohou pokračovat ve vzdělávání na vysoké nebo vyšší odborné škole.
\end{displayquote}

\textbf{SŠ (K)} - úplné střední všeobecné vzdělání (obory kategorie K)

\begin{displayquote}
všeobecná příprava ve 4letých a víceletých gymnáziích je neprofesní a připravuje především pro vysokoškolské nebo vyšší odborné vzdělávání.
\end{displayquote}

\textbf{Obory s~výučním listem}

\textbf{SŠ (H)} - střední odborné vzdělání s~výučním listem (obory kategorie H)

\begin{displayquote}
tradiční učební obory s~tříletou přípravou ve středních odborných učilištích. Po získání výučního listu lze pokračovat navazujícím nástavbovým studiem a získat i maturitu.
\end{displayquote}

\textbf{SŠ (E)} - nižší střední odborné vzdělání (obory kategorie E)

\begin{displayquote}
studium je tříleté nebo dvouleté, výstupem je výuční list. Obory mají nižší nároky v~oblasti všeobecného i obecně odborného vzdělání a jsou určeny především pro žáky se speciálními vzdělávacími potřebami, např. pro absolventy dřívějších speciálních základních škol a žáky, kteří ukončili povinnou školní docházku v~nižším než 9. ročníku základní školy. Obory připravují pro výkon jednoduchých prací v~rámci dělnických povolání a ve službách.
\end{displayquote}

\section{Vymezení dokumentů}

Pro prozkoumání, jak jsou témata zařazena ve vzdělání, pro nás budou primárním zdrojem zejména Rámcové vzdělávací plány (dále RVP). Je nutné se dívat na jejich návaznost, podobnosti a rozdíly na různých stupních a zaměřeních vzdělávání.

Konkrétně se tedy podíváme na RVP pro základní vzdělávání, které definují vzdělávací oblast Informatiky a její cíle, a dále budeme zkoumat RVP pro gymnázia a vybrané odborné školy se zaměřením na Informatiku a Informační technologie. 

Kromě toho rozebereme aktuální plány na revizi oblasti Informatiky a ICT, která může přinášet změny i v~této oblasti, a zároveň ukazovat na možné trendy v~pojetí vzdělávání v~oblasti digitálního světa a technologií a s~tím souvisejících témat.

Konkrétně tedy v~kapitole prozkoumáme propojení na následující dokumenty:

\begin{itemize}
\item Rámcové vzdělávací plány
	\begin{itemize}
    \item RVP základní školy \citep{rvp-zs}
    \item RVP-G (kategorie L) \citep{rvp-g}
    \item RVP pro odborné školy - Informační technologie \citep{rvp-it} (kategorie M)
    \item RVP pro odborné školy - Informační služby\citep{rvp-is} (kategorie M)
	\end{itemize}
\item Návrh revizí rámcových vzdělávacích programů v oblasti informatiky a informačních a komunikačních technologií \citep{revize}
\end{itemize}

\section{Rámcové vzdělávací plány}

\subsection{RVP základní školy}

\textbf{Vzdělávací oblast Informatika}

Ve vymezení Cílového zaměření vzdělávací oblasti je v~kontextu tématu důležitý tento bod

\begin{displayquote}
uvědomění si, respektování a zmírnění negativních vlivů moderních informačních a komunikačních technologií na společnost a na zdraví člověka, ke znalosti způsobů prevence a ochrany před zneužitím a omezováním osobní svobody člověka
\end{displayquote}

konkrétně však tento bod není vymezen cíli ani učivem, které by s~oblastí osobních dat a digitální stopy přímo souviselo.

\subsection{RVP-G}

\textbf{Vzdělávací oblast Informatika a informační a komunikační technologie}

Vzdělávací oblast Informatika a informační a komunikační technologie v~RPV pro vyšší stupně vzdělávání navazuje na obast Informatika v~RPV pro základní školy. V~cílovém zaměření vzdělávací oblasti se tedy nachází totožný bod:

\begin{displayquote}
uvědomění si, respektování a zmírnění negativních vlivů moderních informačních a komunikačních technologií na společnost a na zdraví člověka, ke znalosti způsobů prevence a ochrany před zneužitím a omezováním osobní svobody člověka
\end{displayquote}

V~oblasti Zdroje a vyhledávání informací, Komunikace je v~učivu bod

\begin{displayquote}informační etika, legislativa – ochrana autorských práv a osobních údajů
\end{displayquote}

navazující na výstup

\begin{displayquote}
využívá informační a komunikační služby v~souladu s~etickými, bezpečnostními a legislativními požadavky
\end{displayquote}

Alespoň částečně tedy jde o~problematiku osobních údajů, ačkoli více z~pohledu legislativního a etického než z~pohledu soukromí.

V~dalších oblastech se pak téma neobjevuje. 

\subsection{RVP odborné školy - Informační technologie}

V~cílech vzdělávání můžeme najít následující body (zvýraznění vlastní)

\begin{displayquote}
\textbf{neohrožovali svým chováním v~digitálním prostředí sebe, druhé}, ani technologie samotné
\end{displayquote}

\begin{displayquote}
\textbf{uvědomovali si, že technologie ovlivňují společnost, a naopak chápali svou odpovědnost při používání technologií}
\end{displayquote}

Ve výsledcích vzdělávání pak najdeme konkrétní body

\begin{displayquote}
\textbf{chrání} digitální zařízení, digitální obsah i \textbf{osobní údaje} v~digitálním prostředí před poškozením, přepisem/změnou či \textbf{zneužitím}; reaguje na změny v~technologiích ovlivňujících bezpečnost
\end{displayquote}

\begin{displayquote}
s~vědomím souvislostí fyzického a digitálního světa \textbf{vytváří a spravuje jednu či více digitálních identit; kontroluje svou digitální stopu, ať už ji vytváří sám nebo někdo jiný, v~případě potřeby dokáže používat služby internetu anonymně}
\end{displayquote}

\subsection{RVP odborné školy - Informační služby}

V~tomto RVP se žádné body přímo související s~tématikou nenachází.

\section{Návrh revizí RVP v~oblasti Informatiky a informačních a komunikačních technologií}

Aktuálně se pracuje na revizi RVP v~oblasti Informatiky a informačních a komunikačních technologií, je tedy vhodné se podívat, zda tato revize nějak mění zakotvení tématu digitální stopy a osobních dat ve vzdělávání.

Jedno ze základních východisek návrhu revize je rozvoj digitální gramotnosti, v~dokumentu definované jako 

\begin{displayquote}
Digitální gramotností rozumíme soubor digitálních kompetencí (vědomostí, dovedností, postojů, hodnot), které jedinec potřebuje k bezpečnému, sebejistému, kritickému a tvořivému využívání digitálních technologií při práci, při učení, ve volném čase i při svém zapojení do společenského života.
\end{displayquote}

V~oblastech digitální gramotnosti může být pak pro naše téma relevantní bod

\begin{displayquote}
Vnímá a hodnotí potenciál i rizika zapojení digitálních technologií do různých procesů a v různých situacích a podle toho zodpovědně jedná.
\end{displayquote}

Jak se to promítá přímo do očekávaných výstupů můžeme vidět v~tabulce níže:

% TODO: přidat obrázky

Na všechny typech SŠ nacházíme výstup

\begin{displayquote}
\textbf{chrání} digitální zřízení, digitální obsah i \textbf{osobní údaje} v~digitálním prostředí před poškozením či zneužitím
\end{displayquote}

A~pro školy v~kategoriích K, L, M, H dále

\begin{displayquote}
\textbf{kontroluje} svou \textbf{digitální stopu}, ať už ji vytváří sám nebo někdo jiný, dokáže používat služby internetu anonymně
\end{displayquote}

a pro SŠ (E) podobný bod

\begin{displayquote}
buduje svou digitální identitu a zajímá se, jak k~ní přispívají ostatní
\end{displayquote}

\section{Shrnutí}

Téma osobních dat a digitální stopy můžeme v~aktuálních vzdělávacích materiálech najít v~obecném vymezení v~cíli vzdělávací oblastí Informatika (respektive Informatika a informační a komunikační technologie) v~bodě

\begin{displayquote}
uvědomění si, respektování a zmírnění negativních vlivů moderních informačních a komunikačních technologií na společnost a na zdraví člověka, ke znalosti způsobů prevence a ochrany před zneužitím a omezováním osobní svobody člověka
\end{displayquote}

Konkrétní vzdělávací výsledky s~tímto tématem jsme našli pouze u~RVP odborných škol oboru Informační technologie

Zároveň se však toto téma výrazněji objevuje v~návrhu revize, a tedy můžeme předpokládat, že se do RVP (a následně ŠVP jednotlivých škol) bude více promítat.

Vznik materiálů a prostředí pro zařazení této problematiky do výuky tedy považuji za užitečný a do budoucna nutný.